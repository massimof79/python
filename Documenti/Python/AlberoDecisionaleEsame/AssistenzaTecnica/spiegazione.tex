\documentclass[a4paper,12pt]{article}
\usepackage[italian]{babel}
\usepackage[utf8]{inputenc}
\usepackage{geometry}
\usepackage{hyperref}
\geometry{margin=2.5cm}

\title{Descrizione di un Sistema di Classificazione della Priorità delle Richieste di Assistenza}
\author{}
\date{}

\begin{document}

\maketitle

\section{Obiettivo del sistema}

Il codice implementa un semplice sistema di \textbf{Machine Learning} per stimare automaticamente la \textbf{priorità di una richiesta di assistenza tecnica}.  
L'idea di base è utilizzare dati storici relativi a richieste precedenti per addestrare un modello in grado di prevedere la priorità di nuove segnalazioni.

Il sistema è pensato per simulare uno scenario realistico di \emph{service desk IT}, in cui le richieste possono riguardare problemi software, hardware o di rete, con diverso impatto sugli utenti e sui servizi.

\section{Struttura generale del programma}

Il programma è organizzato in quattro parti principali:

\begin{itemize}
    \item Preparazione dei dati
    \item Addestramento del modello
    \item Previsione su nuovi dati inseriti da tastiera
    \item Menu interattivo per l’utente
\end{itemize}

Sono inoltre presenti due variabili globali:

\begin{itemize}
    \item \texttt{modello}: conterrà il classificatore addestrato
    \item \texttt{colonne\_modello}: memorizza la struttura delle colonne generate durante la codifica dei dati, necessaria per effettuare previsioni coerenti
\end{itemize}

\section{Preparazione dei dati}

La funzione \texttt{carica\_e\_prepara\_dati(percorso\_csv)} si occupa di:

\begin{enumerate}
    \item Caricare un file CSV contenente richieste di assistenza passate.
    \item Separare la variabile target \texttt{Priorità} (cioè ciò che vogliamo prevedere) dalle altre colonne descrittive.
    \item Trasformare le variabili categoriali in variabili numeriche tramite \textbf{One-Hot Encoding}.
\end{enumerate}

Il One-Hot Encoding converte ogni categoria testuale in una colonna binaria (0/1).  
Ad esempio, la variabile \emph{Tipo\_problema} può generare colonne come:

\begin{itemize}
    \item Tipo\_problema\_software
    \item Tipo\_problema\_hardware
    \item Tipo\_problema\_rete
\end{itemize}

Questo passaggio è necessario perché gli algoritmi di Machine Learning lavorano su dati numerici.

\section{Addestramento del modello}

La funzione \texttt{addestra\_modello()} realizza la fase di apprendimento.

\subsection{Suddivisione dei dati}

Il dataset viene suddiviso in:

\begin{itemize}
    \item \textbf{Training set} (70\%): usato per addestrare il modello
    \item \textbf{Test set} (30\%): usato per valutarne le prestazioni
\end{itemize}

\subsection{Algoritmo utilizzato}

Il modello scelto è un \textbf{Decision Tree Classifier} (albero decisionale), configurato con:

\begin{itemize}
    \item criterio \texttt{gini}, per misurare la qualità delle divisioni
    \item profondità massima pari a 4, per evitare un eccessivo adattamento ai dati (overfitting)
\end{itemize}

L’albero decisionale apprende regole del tipo:

\begin{quote}
Se l’impatto è alto e l’urgenza è alta allora la priorità è elevata.
\end{quote}

\subsection{Valutazione}

Dopo l’addestramento, il modello effettua previsioni sul test set.  
Le prestazioni vengono misurate tramite l’\textbf{accuratezza}, cioè la percentuale di previsioni corrette.

\section{Previsione su nuove richieste}

La funzione \texttt{effettua\_previsione()} permette all’utente di inserire manualmente i dati di una nuova richiesta:

\begin{itemize}
    \item Tipo di problema
    \item Numero di utenti coinvolti
    \item Impatto sul servizio
    \item Urgenza dichiarata
\end{itemize}

Anche questi dati vengono codificati con One-Hot Encoding.  
Poiché le colonne generate devono essere identiche a quelle usate in addestramento, il codice riallinea la struttura usando \texttt{colonne\_modello}. Eventuali colonne mancanti vengono riempite con 0.

Il modello produce quindi la \textbf{priorità prevista} per la richiesta.

\section{Interfaccia utente}

La funzione \texttt{menu()} implementa un semplice menu testuale che consente di:

\begin{itemize}
    \item Addestrare il modello
    \item Effettuare una previsione
    \item Uscire dal programma
\end{itemize}

Il sistema è quindi pensato come un piccolo applicativo didattico interattivo, utile per comprendere l’intero flusso di un progetto di Machine Learning:

\begin{center}
Dati $\rightarrow$ Preparazione $\rightarrow$ Addestramento $\rightarrow$ Valutazione $\rightarrow$ Previsione
\end{center}

\section{Conclusione}

Il codice dimostra in modo completo e compatto come costruire un sistema di classificazione supervisionata applicato a un contesto realistico.  
Oltre agli aspetti algoritmici, evidenzia due concetti fondamentali:

\begin{itemize}
    \item la necessità di trasformare correttamente i dati categoriali
    \item l'importanza di mantenere coerenza tra fase di addestramento e fase di previsione
\end{itemize}

Si tratta di un esempio didattico efficace per introdurre alberi decisionali, preprocessing dei dati e valutazione delle prestazioni di un modello predittivo.

\end{document}
