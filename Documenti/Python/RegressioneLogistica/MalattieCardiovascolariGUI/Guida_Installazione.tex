\documentclass[11pt,a4paper]{article}
\usepackage[utf8]{inputenc}
\usepackage[T1]{fontenc}
\usepackage[italian]{babel}
\usepackage{graphicx}
\usepackage{listings}
\usepackage{xcolor}
\usepackage{hyperref}
\usepackage{geometry}
\usepackage{fancyhdr}
\usepackage{tcolorbox}
\usepackage{enumitem}

\geometry{margin=2.5cm}

\definecolor{codegray}{gray}{0.95}
\definecolor{codeblue}{RGB}{52, 152, 219}
\definecolor{codecomment}{RGB}{127, 127, 127}
\definecolor{codestring}{RGB}{255, 140, 0}
\definecolor{warningbg}{RGB}{255, 243, 205}
\definecolor{warningborder}{RGB}{255, 193, 7}
\definecolor{infobg}{RGB}{209, 236, 241}
\definecolor{infoborder}{RGB}{23, 162, 184}
\definecolor{successbg}{RGB}{212, 237, 218}
\definecolor{successborder}{RGB}{40, 167, 69}

\lstset{
	backgroundcolor=\color{codegray},
	basicstyle=\ttfamily\small,
	frame=single,
	keywordstyle=\color{codeblue}\bfseries,
	commentstyle=\color{codecomment}\itshape,
	stringstyle=\color{codestring},
	breaklines=true,
	showstringspaces=false,
	numbers=none,
}

\tcbuselibrary{skins,breakable}

\newtcolorbox{warningbox}{
	colback=warningbg,
	colframe=warningborder,
	leftrule=4pt,
	arc=0pt,
	outer arc=0pt,
	boxrule=0pt,
	breakable
}

\newtcolorbox{infobox}{
	colback=infobg,
	colframe=infoborder,
	leftrule=4pt,
	arc=0pt,
	outer arc=0pt,
	boxrule=0pt,
	breakable
}

\newtcolorbox{successbox}{
	colback=successbg,
	colframe=successborder,
	leftrule=4pt,
	arc=0pt,
	outer arc=0pt,
	boxrule=0pt,
	breakable
}

\hypersetup{
	colorlinks=true,
	linkcolor=blue,
	urlcolor=codeblue,
	pdfborder={0 0 0}
}

\pagestyle{fancy}
\fancyhf{}
\fancyhead[L]{Sistema di Predizione del Rischio Cardiovascolare}
\fancyhead[R]{\thepage}
\fancyfoot[C]{Guida all'Installazione e all'Avvio - Versione 1.0}

\title{\textbf{Sistema di Predizione del Rischio Cardiovascolare}\\
	\large Guida Completa all'Installazione e all'Avvio}
\author{Prof. Fedeli Massimo}
\date{Versione 1.0 - \today}

\begin{document}
	
	\maketitle
	\thispagestyle{empty}
	
	\begin{center}
		\textit{Basato su Regressione Logistica}
	\end{center}
	
	\vspace{1cm}
	
	\tableofcontents
	\newpage
	
	\section{Requisiti di Sistema}
	
	\subsection{Software Necessario}
	
	\begin{itemize}[leftmargin=*]
		\item \textbf{Python 3.7 o superiore} - Linguaggio di programmazione per il backend
		\item \textbf{pip} - Gestore pacchetti Python (solitamente incluso con Python)
		\item \textbf{Browser Web moderno} - Chrome, Firefox, Safari o Edge
		\item \textbf{Editor di testo} - VS Code, Sublime Text, o qualsiasi editor
	\end{itemize}
	
	\subsection{Verifica Installazione Python}
	
	Per verificare se Python è installato correttamente, esegui nel terminale:
	
	\begin{lstlisting}[language=bash]
		python --version
		# oppure
		python3 --version
	\end{lstlisting}
	
	\begin{infobox}
		\textbf{Nota:} Se Python non è installato, scaricalo da \url{https://www.python.org/downloads/}
	\end{infobox}
	
	\section{Installazione delle Dipendenze}
	
	Apri il terminale o prompt dei comandi e installa i pacchetti necessari:
	
	\begin{lstlisting}[language=bash]
		pip install flask flask-cors pandas scikit-learn
	\end{lstlisting}
	
	\begin{successbox}
		\textbf{Pacchetti installati:}
		\begin{itemize}
			\item \textbf{flask} - Framework web per il server API
			\item \textbf{flask-cors} - Gestione Cross-Origin Resource Sharing
			\item \textbf{pandas} - Manipolazione e analisi dati
			\item \textbf{scikit-learn} - Libreria di Machine Learning
		\end{itemize}
	\end{successbox}
	
	\section{Struttura del Progetto}
	
	Crea una cartella per il progetto e organizza i file come segue:
	
	\begin{lstlisting}
		cardiovascular-prediction/
		|
		|-- app.py              # Server Flask (Backend)
		|-- index.html          # Interfaccia web (Frontend)
		|-- README.md           # Documentazione (opzionale)
	\end{lstlisting}
	
	\section{Configurazione e Avvio del Backend}
	
	\subsection{Passo 1: Crea il file app.py}
	
	Crea un nuovo file chiamato \texttt{app.py} e copia il seguente codice:
	
	\begin{lstlisting}[language=Python]
		"""
		API Flask per Predizione Rischio Cardiovascolare
		"""
		
		from flask import Flask, request, jsonify
		from flask_cors import CORS
		import pandas as pd
		from sklearn.model_selection import train_test_split
		from sklearn.preprocessing import StandardScaler
		from sklearn.linear_model import LogisticRegression
		
		app = Flask(__name__)
		CORS(app)
		
		print("Caricamento del modello...")
		
		url = "https://archive.ics.uci.edu/ml/machine-learning-databases/heart-disease/processed.cleveland.data"
		
		colonne = [
		'age', 'sex', 'cp', 'trestbps', 'chol', 'fbs',
		'restecg', 'thalach', 'exang', 'oldpeak',
		'slope', 'ca', 'thal', 'target'
		]
		
		df = pd.read_csv(url, names=colonne, na_values='?')
		df = df.dropna()
		df['target'] = (df['target'] > 0).astype(int)
		
		X = df.drop('target', axis=1)
		y = df['target']
		
		X_train, X_test, y_train, y_test = train_test_split(
		X, y, test_size=0.2, random_state=42, stratify=y
		)
		
		scaler = StandardScaler()
		X_train = scaler.fit_transform(X_train)
		X_test = scaler.transform(X_test)
		
		model = LogisticRegression(max_iter=1000)
		model.fit(X_train, y_train)
		
		print("Modello addestrato e pronto!")
		
		@app.route('/predict', methods=['POST'])
		def predict():
		try:
		data = request.json
		
		paziente = pd.DataFrame([{
			'age': data['age'],
			'sex': data['sex'],
			'cp': data['cp'],
			'trestbps': data['trestbps'],
			'chol': data['chol'],
			'fbs': data['fbs'],
			'restecg': data['restecg'],
			'thalach': data['thalach'],
			'exang': data['exang'],
			'oldpeak': data['oldpeak'],
			'slope': data['slope'],
			'ca': data['ca'],
			'thal': data['thal']
		}])
		
		paziente_scaled = scaler.transform(paziente)
		
		predizione = model.predict(paziente_scaled)[0]
		probabilita = model.predict_proba(paziente_scaled)[0][1]
		
		return jsonify({
			'prediction': int(predizione),
			'probability': float(probabilita),
			'risk': 'high' if predizione == 1 else 'low',
			'message': 'A RISCHIO' if predizione == 1 else 'SANO'
		})
		
		except Exception as e:
		return jsonify({'error': str(e)}), 400
		
		@app.route('/health', methods=['GET'])
		def health():
		return jsonify({'status': 'ok', 'model': 'Logistic Regression'})
		
		if __name__ == '__main__':
		app.run(debug=True, port=5000)
	\end{lstlisting}
	
	\subsection{Passo 2: Avvia il server}
	
	Nel terminale, dalla cartella del progetto, esegui:
	
	\begin{lstlisting}[language=bash]
		python app.py
	\end{lstlisting}
	
	\begin{successbox}
		\textbf{Output atteso:}
		\begin{lstlisting}
			Caricamento del modello...
			Modello addestrato e pronto!
			* Running on http://127.0.0.1:5000
			* Restarting with stat
			* Debugger is active!
		\end{lstlisting}
	\end{successbox}
	
	\section{Configurazione e Avvio del Frontend}
	
	\subsection{Passo 1: Crea il file index.html}
	
	Il codice HTML completo è già stato fornito. Salvalo come \texttt{index.html} nella stessa cartella del progetto.
	
	\subsection{Passo 2: Apri il file HTML}
	
	\begin{itemize}
		\item Fai doppio clic su \texttt{index.html}
		\item Oppure aprilo con il browser (File → Apri File)
	\end{itemize}
	
	\section{Test del Sistema}
	
	\subsection{Test Rapido con Dati Precompilati}
	
	\begin{enumerate}
		\item Il form è già precompilato con i dati di un paziente di esempio
		\item Clicca direttamente su \textbf{``Calcola Rischio''}
		\item Dovresti vedere il risultato della predizione con la probabilità
	\end{enumerate}
	
	\subsection{Parametri del Paziente di Esempio}
	
	\begin{table}[h]
		\centering
		\begin{tabular}{|l|l|l|}
			\hline
			\textbf{Parametro} & \textbf{Valore} & \textbf{Descrizione} \\
			\hline
			Età & 60 & Anni \\
			\hline
			Sesso & Maschile & 1 = M, 0 = F \\
			\hline
			Tipo Dolore Toracico & Angina Tipica & 0-3 \\
			\hline
			Pressione Riposo & 150 & mm Hg \\
			\hline
			Colesterolo & 250 & mg/dl \\
			\hline
			Glicemia Digiuno & Sì (maggiore 120) & 1 = Sì, 0 = No \\
			\hline
			FC Massima & 130 & battiti/min \\
			\hline
			Oldpeak & 2.3 & Depressione ST \\
			\hline
		\end{tabular}
		\caption{Parametri paziente di test}
	\end{table}
	
	\section{Risoluzione Problemi Comuni}
	
	\subsection{Problema 1: Errore di connessione al server}
	
	\begin{warningbox}
		\textbf{Errore:} ``Assicurati che il server Python sia in esecuzione''
		
		\textbf{Soluzione:}
		\begin{itemize}
			\item Verifica che il server Flask sia avviato (controlla il terminale)
			\item Assicurati che sia in esecuzione su \texttt{http://localhost:5000}
			\item Se usi una porta diversa, modifica \texttt{API\_URL} nell'HTML
		\end{itemize}
	\end{warningbox}
	
	\subsection{Problema 2: Modulo non trovato}
	
	\begin{warningbox}
		\textbf{Errore:} ``ModuleNotFoundError: No module named 'flask'''
		
		\textbf{Soluzione:}
		\begin{lstlisting}[language=bash]
			pip install flask flask-cors pandas scikit-learn
		\end{lstlisting}
	\end{warningbox}
	
	\subsection{Problema 3: Errore CORS nel browser}
	
	\begin{warningbox}
		\textbf{Soluzione:}
		\begin{itemize}
			\item Verifica che \texttt{flask-cors} sia installato
			\item Controlla che \texttt{CORS(app)} sia presente in app.py
		\end{itemize}
	\end{warningbox}
	
	\subsection{Problema 4: Porta occupata}
	
	\begin{warningbox}
		\textbf{Soluzione:}
		
		La porta 5000 potrebbe essere occupata. Modifica la porta in app.py:
		
		\begin{lstlisting}[language=Python]
			app.run(debug=True, port=5001)  # Usa 5001 invece di 5000
		\end{lstlisting}
		
		E aggiorna l'URL nell'HTML:
		
		\begin{lstlisting}
			const API_URL = 'http://localhost:5001/predict';
		\end{lstlisting}
	\end{warningbox}
	
	\section{Informazioni sul Modello}
	
	\begin{infobox}
		\subsection*{Algoritmo Utilizzato}
		\textbf{Regressione Logistica} - Un algoritmo di classificazione binaria che stima la probabilità che un paziente sia a rischio di malattia cardiovascolare.
		
		\subsection*{Dataset}
		UCI Machine Learning Repository - Cleveland Heart Disease Database
		
		297 pazienti dopo la pulizia dei dati mancanti
		
		\subsection*{Preprocessing}
		\begin{itemize}
			\item Standardizzazione delle feature (media=0, std=1)
			\item Rimozione valori mancanti
			\item Split 80/20 train/test stratificato
		\end{itemize}
	\end{infobox}
	
	\section{Utilizzo del Sistema}
	
	\begin{enumerate}
		\item Inserisci i dati clinici del paziente nel form
		\item Clicca su ``Calcola Rischio''
		\item Il sistema mostrerà:
		\begin{itemize}
			\item Classificazione: A RISCHIO o SANO
			\item Probabilità in percentuale
			\item Raccomandazioni cliniche
		\end{itemize}
	\end{enumerate}
	
	\vspace{1cm}
	
	\begin{center}
		\rule{\textwidth}{0.4pt}
		
		\vspace{0.5cm}
		
		\textbf{Sistema di Predizione del Rischio Cardiovascolare}
		
		Basato su Machine Learning con Regressione Logistica
		
		Per supporto tecnico o domande, consultare la documentazione di scikit-learn e Flask
		
	\end{center}
	
\end{document}