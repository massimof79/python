\documentclass[a4paper,12pt]{article}

\usepackage[italian]{babel}
\usepackage[utf8]{inputenc}
\usepackage{geometry}
\usepackage{hyperref}
\usepackage{listings}
\usepackage{xcolor}

\geometry{margin=2.5cm}

\title{Applicazione di Machine Learning con Regressione Logistica\\
Stima del Rischio di Diabete}
\author{Prof. Fedeli Massimo}
\date{}

\lstset{
    basicstyle=\ttfamily\small,
    breaklines=true,
    frame=single
}

\begin{document}

\maketitle

\section{Obiettivo del progetto}

In questa esercitazione realizziamo una piccola applicazione di intelligenza artificiale che utilizza un modello di regressione logistica per stimare la probabilità che una persona sia affetta da diabete.

L'applicazione:
\begin{itemize}
    \item usa un dataset medico reale;
    \item addestra un modello di machine learning;
    \item salva il modello su file;
    \item mette il modello a disposizione tramite una pagina web in cui l’utente inserisce i propri dati.
\end{itemize}

\section{Il dataset}

Il modello viene addestrato usando un dataset pubblico contenente informazioni cliniche di diverse pazienti.  
Ogni riga del dataset rappresenta una persona, descritta da alcune misure mediche.

Le principali variabili sono:

\begin{itemize}
    \item \textbf{Pregnancies} – Numero di gravidanze
    \item \textbf{Glucose} – Livello di glucosio nel sangue
    \item \textbf{BloodPressure} – Pressione sanguigna
    \item \textbf{SkinThickness} – Spessore della plica cutanea
    \item \textbf{Insulin} – Livello di insulina
    \item \textbf{BMI} – Indice di massa corporea
    \item \textbf{DiabetesPedigreeFunction} – Indicatore di familiarità
    \item \textbf{Age} – Età
    \item \textbf{Outcome} – 1 se la persona ha il diabete, 0 altrimenti
\end{itemize}

L’ultima colonna (Outcome) è quella che il modello deve imparare a prevedere.

\section{Fase 1: Addestramento del modello}

Il file \texttt{train\_model.py} serve a creare e addestrare il modello.

\subsection{Caricamento dei dati}

\begin{lstlisting}[language=Python]
df = pd.read_csv("diabetes.csv")
\end{lstlisting}

Qui leggiamo il file CSV che contiene il dataset e lo memorizziamo in una struttura dati chiamata DataFrame.

\subsection{Separazione tra input e output}

\begin{lstlisting}[language=Python]
X = df[features]
y = df["Outcome"]
\end{lstlisting}

\begin{itemize}
    \item \texttt{X} contiene le informazioni di ingresso (età, glucosio, ecc.)
    \item \texttt{y} contiene la risposta corretta (diabete sì/no)
\end{itemize}

\subsection{Divisione in training e test}

I dati vengono divisi in:
\begin{itemize}
    \item Training set (80\%) per insegnare al modello
    \item Test set (20\%) per verificare se ha imparato correttamente
\end{itemize}

\subsection{Standardizzazione e modello}

La pipeline contiene due passaggi:
\begin{enumerate}
    \item StandardScaler – normalizza i dati
    \item LogisticRegression – modello matematico che stima una probabilità tra 0 e 1
\end{enumerate}

\subsection{Addestramento}

Il modello osserva i dati di training e impara a collegare i valori clinici alla presenza o meno del diabete.

\subsection{Salvataggio del modello}

\begin{lstlisting}[language=Python]
joblib.dump(pipeline, "modello_diabete.pkl")
\end{lstlisting}

Il modello viene salvato su file, così potrà essere riutilizzato nell'applicazione web senza doverlo riaddestrare ogni volta.

\section{Che cos’è il file \texttt{.pkl}}

Il file con estensione \texttt{.pkl} è un file in formato \textbf{Pickle}, un formato usato da Python per salvare oggetti su disco.

In questo progetto, l’oggetto salvato è l’intera pipeline di machine learning, che include:
\begin{itemize}
    \item i parametri calcolati dallo StandardScaler
    \item i pesi appresi dal modello di regressione logistica
\end{itemize}

Salvare il modello in formato Pickle significa memorizzare su file lo stato interno del modello dopo l’addestramento. In questo modo:

joblib è una libreria Python usata per memorizzare su disco oggetti anche complessi, come modelli di machine learning che contengono molti parametri numerici.

dump è la funzione che effettua il salvataggio. In pratica “impacchetta” un oggetto Python e lo scrive in un file.

pipeline è l’oggetto che vogliamo salvare. Non è solo il modello di regressione lineare, ma tutta la pipeline, cioè:

lo StandardScaler con i parametri calcolati sui dati di training (media e deviazione standard),

il modello LinearRegression con i coefficienti che ha imparato.


\begin{itemize}
    \item non è necessario riaddestrare il modello ogni volta che si avvia il programma;
    \item il modello può essere caricato rapidamente e usato per fare nuove previsioni.
\end{itemize}

Il caricamento avviene con:

\begin{lstlisting}[language=Python]
model = joblib.load("modello_diabete.pkl")
\end{lstlisting}

A questo punto il modello è pronto per ricevere nuovi dati e calcolare la probabilità di diabete.

\section{Fase 2: Applicazione Web con Flask}

Il file \texttt{app.py} crea un piccolo server web che permette agli utenti di usare il modello tramite una pagina web.

Quando l’utente inserisce i dati:
\begin{enumerate}
    \item i valori vengono inviati al server;
    \item il server li passa al modello caricato dal file \texttt{.pkl};
    \item il modello restituisce una probabilità;
    \item il risultato viene mostrato nella pagina.
\end{enumerate}

\end{document}
