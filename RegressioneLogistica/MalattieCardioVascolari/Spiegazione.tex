\documentclass{beamer}

\usepackage[italian]{babel}
\usepackage[utf8]{inputenc}
\usepackage[T1]{fontenc}
\usepackage{graphicx}

\usetheme{Madrid}
\usecolortheme{default}

\title[Sistema di Predizione del Rischio Cardiovascolare]
{Sistema di Predizione del Rischio Cardiovascolare}
\subtitle{Algoritmo di Regressione Logistica}
\author{Prof. Fedeli Massimo}
\institute{IIS Fermi Sacconi Cpia di Ascoli Piceno}
\date{Tutti i diritti riservati}

\begin{document}
	
	% Slide titolo
	\begin{frame}
		\titlepage
	\end{frame}
	
	% Slide 1
	\begin{frame}{Obiettivo del sistema}
		Il sistema realizza una predizione del rischio cardiovascolare basata su tecniche di Machine Learning supervisionato.
		
		\medskip
		L’obiettivo principale è \textbf{classificare} ciascun paziente in una delle due categorie:
		\begin{itemize}
			\item paziente sano
			\item paziente a rischio di malattia cardiaca
		\end{itemize}
		
		\medskip
		La classificazione viene effettuata tramite un modello di regressione logistica addestrato su dati clinici reali.
	\end{frame}
	
	% Slide 2
	\begin{frame}{Dataset e caratteristiche}
		Il punto di partenza è il caricamento di un \textbf{dataset} contenente informazioni cliniche e anagrafiche dei pazienti.
		
		\medskip
		Tra le principali variabili considerate:
		\begin{itemize}
			\item età
			\item sesso
			\item pressione sanguigna
			\item colesterolo
			\item risultati di test diagnostici
		\end{itemize}
		
		\medskip
		I dati vengono letti direttamente da una sorgente online e organizzati in un DataFrame.
	\end{frame}
	
	% Slide 3
	\begin{frame}{Pulizia e trasformazione dei dati}
		Nel dataset originale sono presenti valori mancanti, indicati tramite un simbolo speciale.
		
		\medskip
		La fase di \textbf{preprocessing} prevede:
		\begin{itemize}
			\item conversione dei valori mancanti in valori nulli
			\item eliminazione delle osservazioni incomplete
		\end{itemize}
		
		\medskip
		Questa scelta, pur semplice, risulta adeguata per una versione didattica dell’algoritmo.
	\end{frame}
	
	% Slide 4
	\begin{frame}{Definizione del target}
		Nel dataset originale il grado di malattia è espresso mediante più valori interi (0,1,2,3,4).
		
		\medskip
		Il problema viene semplificato trasformando il target in forma binaria:
		\begin{itemize}
			\item assenza di malattia (0)
			\item presenza di malattia (1)
		\end{itemize}
		
		\medskip
		Questa trasformazione rende il problema più adatto alla regressione logistica e più semplice da interpretare.
	\end{frame}
	
	% Slide 5
	\begin{frame}{Suddivisione del dataset}
		La preparazione dei dati prevede la separazione tra:
		\begin{itemize}
			\item variabili di input (feature cliniche)
			\item variabile di output (stato di salute)
		\end{itemize}
		
		\medskip
		Il dataset viene suddiviso in:
		\begin{itemize}
			\item insieme di addestramento
			\item insieme di test
		\end{itemize}
		
		\medskip
		La suddivisione è stratificata, per mantenere proporzioni simili di pazienti sani e malati.
	\end{frame}
	
	% Slide 6
	\begin{frame}{Standardizzazione delle feature}
		Le variabili cliniche presentano scale molto diverse tra loro.
		
		\medskip
		Prima dell’addestramento viene applicata la standardizzazione:
		\begin{itemize}
			\item media nulla
			\item deviazione standard unitaria
		\end{itemize}
		
		\medskip
		Questo passaggio è fondamentale per la \textbf{regressione logistica}, che è sensibile alle differenze di scala tra le variabili.
	\end{frame}
	
	% Slide 7
	\begin{frame}{Addestramento del modello}
		Il modello utilizzato è una regressione logistica per problemi di classificazione binaria.
		
		\medskip
		Durante l’addestramento il modello:
		\begin{itemize}
			\item apprende un insieme di pesi
			\item quantifica l’influenza di ciascuna variabile clinica
		\end{itemize}
		
		\medskip
		L’output del modello è la probabilità che un paziente appartenga alla classe “a rischio”.
	\end{frame}
	
	% Slide 8
	\begin{frame}{Valutazione delle prestazioni}
		Il modello viene valutato sui \textbf{dati di test}, non utilizzati durante l’addestramento.
		
		\medskip
		Le previsioni vengono confrontate con i valori reali per calcolare l’accuratezza.
		
		\medskip
		L’accuratezza rappresenta la percentuale di classificazioni corrette e fornisce una prima indicazione dell’efficacia del sistema, pur non esaurendo tutte le possibili metriche clinicamente rilevanti.
	\end{frame}
	
\end{document}
