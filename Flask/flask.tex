\documentclass[12pt,a4paper]{article}
\usepackage[utf8]{inputenc}
\usepackage[italian]{babel}
\usepackage{amsmath}
\usepackage{graphicx}
\usepackage{listings}
\usepackage{xcolor}
\usepackage{hyperref}
\usepackage{geometry}
\geometry{margin=2.5cm}

\definecolor{codegray}{gray}{0.95}
\definecolor{myblue}{RGB}{0, 102, 204}

\lstset{
	backgroundcolor=\color{codegray},
	basicstyle=\ttfamily\footnotesize,
	frame=single,
	keywordstyle=\color{blue}\bfseries,
	commentstyle=\color{gray},
	stringstyle=\color{orange},
	breaklines=true,
	postbreak=\mbox{\textcolor{red}{$\hookrightarrow$}\space},
	language=Python,
	morekeywords={app, route, request, jsonify, render_template}
}

\title{Flask: Esecuzione di Python lato Server}
\author{Prof. Fedeli Massimo}
\date{\today}

\begin{document}
	
	\maketitle
	
	\tableofcontents
	\newpage
	
	\section{Introduzione}
	\textbf{Flask} è un \textit{micro-framework} web per Python, leggero, flessibile e ideale per creare applicazioni web o API RESTful.  
	Permette di eseguire codice Python lato server, rispondendo a richieste HTTP da parte di client (browser, app, altri servizi).
	
	\section{Caratteristiche Principali}
	\begin{itemize}
		\item \textbf{Leggero}: solo il necessario, niente overhead.
		\item \textbf{Modulare}: si espande con estensioni (es. Flask-CORS, Flask-SQLAlchemy).
		\item \textbf{Python puro}: niente sintassi aggiuntiva o generatori di codice.
		\item \textbf{Werkzeug-based}: usa un server WSGI di sviluppo robusto.
	\end{itemize}
	
	\section{Installazione}
	\begin{lstlisting}
		pip install flask
	\end{lstlisting}
	
	\section{Struttura Minima di un'App Flask}
	\begin{lstlisting}
		from flask import Flask
		
		app = Flask(__name__)
		
		@app.route('/')
		def home():
		return "Ciao da Flask!"
		
		if __name__ == '__main__':
		app.run(debug=True)
	\end{lstlisting}
	
	\section{Routing e Decoratori}
	\subsection{Route Statiche}
	\begin{lstlisting}
		@app.route('/about')
		def about():
		return "Pagina About"
	\end{lstlisting}
	
	\subsection{Route con Parametri}
	\begin{lstlisting}
		@app.route('/user/<nome>')
		def user(nome):
		return f"Ciao, {nome}!"
	\end{lstlisting}
	
	\subsection{Metodi HTTP}
	\begin{lstlisting}
		@app.route('/api/data', methods=['GET', 'POST'])
		def data():
		if request.method == 'POST':
		return jsonify({"ricevuto": request.json})
		return jsonify({"messaggio": "Dati disponibili"})
	\end{lstlisting}
	
	\section{Gestione delle Richieste}
	\subsection{JSON (API)}
	\begin{lstlisting}
		from flask import request, jsonify
		
		@app.route('/predict', methods=['POST'])
		def predict():
		data = request.json
		risultato = {"predizione": data["valore"] * 2}
		return jsonify(risultato)
	\end{lstlisting}
	
	\subsection{Form HTML}
	\begin{lstlisting}
		@app.route('/form', methods=['GET', 'POST'])
		def form():
		if request.method == 'POST':
		nome = request.form['nome']
		return f"Ciao, {nome}!"
		return '''
		<form method="post">
		Nome: <input name="nome">
		<input type="submit">
		</form>
		'''
	\end{lstlisting}
	
	\section{Template HTML (Jinja2)}
	\begin{lstlisting}
		from flask import render_template
		
		@app.route('/hello/<nome>')
		def hello(nome):
		return render_template('hello.html', nome=nome)
	\end{lstlisting}
	
	\textbf{File: \texttt{templates/hello.html}}
	\begin{lstlisting}
		<!doctype html>
		<html>
		<body>
		<h1>Ciao, {{ nome }}!</h1>
		</body>
		</html>
	\end{lstlisting}
	
	\section{CORS: Accesso da Frontend}
	\begin{lstlisting}
		pip install flask-cors
	\end{lstlisting}
	\begin{lstlisting}
		from flask_cors import CORS
		
		app = Flask(__name__)
		CORS(app)  # Abilita CORS su tutte le route
	\end{lstlisting}
	
	\section{Esempio Completo: API RESTful}
	\begin{lstlisting}
		from flask import Flask, request, jsonify
		from flask_cors import CORS
		
		app = Flask(__name__)
		CORS(app)
		
		@app.route('/api/somma', methods=['POST'])
		def somma():
		data = request.json
		a = data.get('a', 0)
		b = data.get('b', 0)
		return jsonify({"risultato": a + b})
		
		if __name__ == '__main__':
		app.run(debug=True)
	\end{lstlisting}
	
	\textbf{Richiesta curl:}
	\begin{lstlisting}
		curl -X POST http://localhost:5000/api/somma \
		-H "Content-Type: application/json" \
		-d '{"a": 3, "b": 4}'
	\end{lstlisting}
	
	\section{Deploy di Base}
	\subsection{Ambiente di Produzione}
	\begin{itemize}
		\item \textbf{Non usare} il server di sviluppo in produzione.
		\item Usa \textbf{Gunicorn} o \textbf{uWSGI} dietro \textbf{Nginx}.
	\end{itemize}
	
	\begin{lstlisting}
		pip install gunicorn
		gunicorn app:app -b 0.0.0.0:8000
	\end{lstlisting}
	
	\section{Conclusione}
	Flask è lo strumento ideale per:
	\begin{itemize}
		\item Creare API RESTful in Python
		\item Integrare modelli di machine learning in applicazioni web
		\item Prototipare rapidamente servizi backend
		\item Apprendere i concetti fondamentali del web development
	\end{itemize}
	
\end{document}